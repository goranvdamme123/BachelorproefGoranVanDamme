%%=============================================================================
%% Methodologie
%%=============================================================================

\chapter{\IfLanguageName{dutch}{Methodologie}{Methodology}}%
\label{ch:methodologie}

%% TODO: In dit hoofstuk geef je een korte toelichting over hoe je te werk bent
%% gegaan. Verdeel je onderzoek in grote fasen, en licht in elke fase toe wat
%% de doelstelling was, welke deliverables daar uit gekomen zijn, en welke
%% onderzoeksmethoden je daarbij toegepast hebt. Verantwoord waarom je
%% op deze manier te werk gegaan bent.
%% 
%% Voorbeelden van zulke fasen zijn: literatuurstudie, opstellen van een
%% requirements-analyse, opstellen long-list (bij vergelijkende studie),
%% selectie van geschikte tools (bij vergelijkende studie, "short-list"),
%% opzetten testopstelling/PoC, uitvoeren testen en verzamelen
%% van resultaten, analyse van resultaten, ...
%%
%% !!!!! LET OP !!!!!
%%
%% Het is uitdrukkelijk NIET de bedoeling dat je het grootste deel van de corpus
%% van je bachelorproef in dit hoofstuk verwerkt! Dit hoofdstuk is eerder een
%% kort overzicht van je plan van aanpak.
%%
%% Maak voor elke fase (behalve het literatuuronderzoek) een NIEUW HOOFDSTUK aan
%% en geef het een gepaste titel.



De eerste fase zal gewijd worden aan het analyseren van de concrete problemen die zich mogelijks kunnen voordoen bij het behouden van de originele firewall, dit kan gaan over technische en non-technische aspecten die mogelijks de dienstverlening binnen de productie kan verstoren en de financiële schade die VPK daar mogelijk zou kunnen door oplopen. Ook zal er worden gekeken hoe het implementeren van die nieuwe firewall toepassing dit probleem zouden kunnen beheren en voorkomen. Hiervoor zal er gebruik gemaakt worden van bedrijfsinterne knowledge bases, diverse papers en studies met betrekking tot netwerkbeveiliging. Zo zal er een beter inzicht te verkrijgen zijn over de mogelijke firewall opstellingen en welke strategieën er kunnen worden toegepast voor het beschermen van het ICS met behulp van firewalltoepassingen. Ook zal er worden gekeken naar andere aspecten die mogelijks spelen bij het maken van een doordachte keuze. Dit kan gaan over de ervaringen van het huidige netwerkteam met een bepaalde vendor. Maar mogelijks over de integratie met andere netwerkcomponenten en de mogelijkheid om de firewall centraal te beheren. De geschatte duurtijd van deze fase bedraagt drie dagen die verspreid zullen zijn over drie weken.


Tijdens de tweede fase zal er een interne bevraging zijn bij systeembeheerders en netwerkprofessionals op bedrijfsniveau. Hun jarenlange ervaring omtrent netwerkbeveiliging en het optimaliseren van talloze beveiligingsmaatregelen binnen VPK kan kostbare inzichten opleveren voor het achterhalen van de huidige netwerkopstelling van het bedrijf. Deze informatie zal gebruikt kunnen worden voor het configureren van de firewall zodanig dat deze optimaal geconfigureerd is in het netwerk. Zo kunnen we de schade die een aanval zou kunnen aanrichten zo minimaal mogelijk houden en zal de aanval de dienstverlening zo weinig mogelijk verstoren. Ook kan deze informatie nuttig zijn voor de evaluatie in de vijfde fase. De geschatte duurtijd van deze fase bedraagt drie dagen die verspreid zullen zijn over drie weken.

In de derde fase verwerken we de grote hoeveelheid verzamelde data om een helder overzicht te creëren. Dit maakt het makkelijker om te beslissen of we de huidige firewall opnieuw configureren of een volledig nieuwe implementeren. Deze keuze baseren we op verschillende analyses, zoals een SWOT en RASCI analyse. Hiermee brengen we niet alleen de technische voor- en nadelen in kaart, maar kijken we ook naar welke stakeholders welke taken op zich nemen na de implementatie. Op basis van deze inzichten stellen we een concreet plan op voor eventuele aanpassingen aan de firewall en het netwerk van de productiesite. De geschatte duurtijd van deze fase bedraagt twee dagen die verspreid zullen zijn over twee weken.

In het vierde deel zal er worden toegespitst op het perfectioneren van beveiligingstoepassingen. Hierbij zal er met behulp van het plan dat is opgesteld in de derde fase de huidige firewalltoepassing geherconfigureerd worden. Als er aan de hand van de verschillende analyses blijkt dat een andere firewall dan de huidige firewall een betere 'fit' zou kunnen zijn voor de eisen en noden van VPK dan kan er worden gekozen om gebruik te maken van een compleet nieuwe firewall die ook betere bescherming zal bieden tegen aanvallen op het ICS. Ook zullen er aanpassingen kunnen worden aangebracht aan de rest van het netwerk die de firewall zou kunnen helpen bij het tegenhouden van aanvallen op de ICS. De geschatte duurtijd van deze fase bedraagt drie dagen die verspreid zullen zijn over drie weken.


De vijfde en tevens laatste fase zal bestaan uit het evalueren van de impact van de geherconfigureerde firewall en de voorgestelde aanpassingen aan het netwerk. Er zullen een reeks criteria worden opgesteld waaraan voldaan moet zijn. Deze criteria zullen er voor zorgen dat we duidelijk kunnen aantonen dat mogelijke wijzigingen die we hebben aangebracht het makkelijker maken voor de systeembeheerders om de firewall te beheren en een ze ook een duidelijker inzicht krijgen van welke zaken er juist gebeuren aan de interne kant van de firewall. Daarnaast zal worden beschreven welke specifieke maatregelen het beste kunnen worden genomen om het netwerk en de firewall van het productiebedrijf beter te beschermen tegen aanvallen op de ICS. De geschatte duur van deze fase bedraagt twee dagen, verdeeld over twee weken.  



\label{ch:methodologie}

\section{Analyse van de huidige opstelling}

De eerste fase van dit onderzoek richt zich voornamelijk op het verzamelen van relevante informatie die nodig is voor een weloverwogen keuze van de toekomstige firewallopstelling. Bij de implementatie van IT- of OT-toepassingen binnen een productiesite moet met een groot aantal factoren rekening worden gehouden. De hoogste prioriteit van elke productiesite blijft de continuïteit van de productie. Omdat de implementatie van een nieuwe netwerktopologie verschillende risico’s met zich meebrengt, is het cruciaal om alle mogelijke problemen en scenario’s zorgvuldig in kaart te brengen. Dit voorkomt verstoringen in het productieproces en waarborgt de continuiteit van de productie.
Een verstoring van de IT-dienstverlening binnen een productiesite heeft niet alleen impact op de IT-apparaten van gebruikers, maar ook op de OT toepassingen, zoals PLC’s, sensoren en SCADA-systemen. Deze systemen spelen een belangerijke rol in het aansturen en monitoren van industriële processen. Wanneer de IT infrastructuur faalt, kunnen deze OT systemen niet meer correct functioneren, dit heeft directe gevolgen voor de productieomgeving. Daarnaast is de werking van machines binnen een productiesite sterk afhankelijk van onderlinge communicatie tussen hen. Ze zijn geprogrammeerd om in een nauwkeurige volgorde en met optimale efficiëntie te opereren, zodat de productie soepel verloopt en de supply chain niet wordt onderbroken. Als de verbinding tussen deze machines wegvalt, ontstaan er onvoorziene stilstanden en inefficiënties die de productie ernstig kunnen vertragen of zelfs volledig stilleggen. Dit heeft niet alleen gevolgen voor de interne werking van de productiesite, maar ook voor externe partners en leveranciers. Transportbedrijven, logistieke dienstverleners en andere samenwerkingspartners ervaren hinder wanneer de supply chain verstoord raakt. De expeditieafdeling, die afhankelijk is van verschillende softwaretools en netwerksystemen zoals SAP voor het plannen en coördineren van leveringen, kan haar taken niet efficiënt uitvoeren zonder een stabiele IT omgeving. Dit leidt tot vertragingen in verzendingen, problemen met voorraadbeheer en mogelijk zelfs financiële verliezen door stilstand en gemiste levertermijnen.


\subsection{Problemen met huidige Firewall}
In 2021 heeft VPK de productiesite van Alizay overgenomen van papierproducent Double A. Met deze overname kreeg VPK niet alleen de productiefaciliteit in handen, maar ook het volledige IT- en OT-netwerk dat op deze site in gebruik was. Aangezien Double A gebruikmaakte van andere leveranciers voor netwerkapparatuur en eigen best practices hanteerde voor netwerkbeheer, verschilt de opzet van dit netwerk aanzienlijk van de standaard IT/OT-infrastructuur die VPK binnen al haar andere productiesites gebruikt.
Dit verschil brengt verschillende uitdagingen met zich mee. Het overgenomen netwerk heeft diverse problemen met de stabiliteit, veiligheid en efficiëntie die de IT en OT systemen in gevaar brengen. Compatibiliteitsproblemen tussen de originele infrastructuur van Alizay en de standaarden van VPK maken het lastig om systemen correct te integreren met elkaar.
 
Om te beginnen is het OT-netwerk van deze site beveiligd met een firewall van het merk Sophos. Dit vormt een uitdaging, aangezien VPK standaard gebruikmaakt van Palo Alto firewalls voor de beveiliging van alle andere productiesites die ze beheren. Palo Alto biedt een centrale webinterface, genaamd Strata Cloud Manager, waarmee alle geïmplementeerde firewalls binnen het netwerk op één platform kunnen worden beheerd.

Omdat Sophos-firewalls niet compatibel zijn met deze tool, moeten ze afzonderlijk worden beheerd via een eigen online portaal. Dit leidt tot een versnipperde beheeromgeving, waarin netwerkbeheerders meerdere systemen moeten gebruiken om de verschillende firewalls te beheren. Hierdoor neemt de complexiteit toe en wordt het lastiger om een uniforme beveiligingsstrategie te hanteren.

Een bijkomend voordeel van Strata Cloud Manager is dat het niet alleen een centraal overzicht biedt van alle Palo Alto firewalls, maar ook de mogelijkheid geeft om firewallregels op één plek te beheren en automatisch uit te rollen naar specifieke firewalls binnen het netwerk. Dit maakt het eenvoudiger om wijzigingen consistent door te voeren en minimaliseert de kans op fouten bij handmatige configuraties.
Daarnaast kan het gebruik van verschillende firewallmerken leiden tot inconsistente security policies, verhoogde operationele lasten en een grotere kans op menselijke fouten bij configuraties en monitoring. Om een efficiënter en gestroomlijnder netwerkbeheer te realiseren, zou het migreren naar een uniforme firewalloplossing, zoals Palo Alto, een logische volgende stap zijn.

Daarnaast heeft niemand binnen het huidige netwerkteam van VPK ervaring met het implementeren en configureren van Sophos firewalls. Hoewel dit op dit moment geen direct probleem vormt voor de huidige firewallopstelling, is het wel een belangrijke factor om in overweging te nemen. Op langere termijn kan het gebrek aan expertise leiden tot verschillende uitdagingen, zoals een incorrecte configuratie van de firewalls door onvoldoende kennis van de specifieke implementatie- en beheerprincipes van Sophos.

Een verkeerde configuratie kan resulteren in beveiligingsrisico’s, verminderde netwerkprestaties of zelfs netwerkstoringen. Bovendien kan het ontbreken van interne kennis van het netwerkteam over Sophos hardware ertoe leiden dat het netwerkteam afhankelijk wordt van externe consultants voor ondersteuning en onderhoud, wat extra kosten en vertragingen met zich mee kan brengen. Om deze risico’s te beperken, kan er overwogen worden om over te stappen op een firewalloplossing die beter aansluit bij de bestaande kennis en beheertools binnen VPK zoals een firewall van Palo Alto.



\subsection{Problemen met het huidige OT netwerk}
Een van de grootste uitdagingen binnen het netwerk van de productiesite van Alizay is het gebrek aan documentatie over de bestaande infrastructuur. Het netwerkteam dat verantwoordelijk was voor het onderhoud van het IT- en OT-netwerk voor de overname door VPK heeft geen gedetailleerde documentatie bijgehouden over de netwerkarchitectuur. Dit maakt het beheer van het netwerk moeilijker, omdat er onvoldoende beschikbaar is over de werking en samenstelling ervan. Zonder deze kennis wordt het uitdagend om aanpassingen door te voeren, storingen snel op te lossen en beveiligingsrisico’s effectief te beheersen.
Het OT-netwerk binnen de site van Alizay bestaat uit meerdere generaties componenten die in de loop van tientallen jaren zijn geïmplementeerd. Hierdoor is het netwerk geleidelijk uitgebreid met verschillende subnetwerken die doorheen de jaren zijn toegevoegd, vaak zonder een plan of duidelijke standaardisatie. Dit heeft geleid tot een complexe en gefragmenteerde infrastructuur, waarbij verschillende systemen en protocollen naast elkaar bestaan.
Deze gelaagde netwerkstructuur kan verschillende uitdagingen veroorzaken wanneer wordt besloten de huidige firewallregels te herzien of te optimaliseren. Op dit moment zijn er geen specifieke firewallregels die expliciet bepalen welke data uit de verschillende subnetwerken mag worden doorgelaten of geblokkeerd. Dit betekent dat het netwerk mogelijk meer verkeer toestaat dan strikt noodzakelijk is, wat een potentieel beveiligingsrisico vormt. Tegelijkertijd zorgt het ontbreken van een duidelijke segmentatie van het netwerk ervoor dat ongewenste blokkades kunnen optreden wanneer nieuwe firewallregels worden ingevoerd.
De combinatie van een verouderde netwerkinfrastructuur en het gebrek aan een goed gedefinieerd firewallbeleid kan het risico met zich mee brengen dat bepaalde subnetwerken of systemen onverwacht worden geblokkeerd bij herconfiguratie van de firewall. Dit kan ertoe leiden dat machines tijdelijk uitvallen, wat een negatieve impact heeft op de productieprocessen en de volledige supply chain. In sommige gevallen kan zelfs een kleine onderbreking in de netwerkcommunicatie ervoor zorgen dat een machine niet meer correct functioneert, waardoor productiestilstand ontstaat en de continuiteit van de productie niet gewaarborgd kan worden.
Binnen de verschillende productiesites van VPK wordt gebruikgemaakt van machines afkomstig van diverse externe leveranciers. Deze geavanceerde machines bevatten vaak een combinatie van verschillende technische componenten, zoals PLC’s, sensoren, pneumatische systemen, robotarmen en geautomatiseerde transportsystemen zoals die van Minda. Deze componenten zijn onderling afhankelijk en wisselen continu data uit om productieprocessen op elkaar af te stemmen. Om deze gegevensuitwisseling mogelijk te maken moet er een betrouwbaar netwerk aanwezig zijn die alle verschillende componenten met elkaar verbindt.
In de praktijk betekent dit dat fabrikanten van deze machines vaak genoodzaakt zijn om hun eigen kleine netwerken op te zetten binnen de productiesite. Dit gebeurt met behulp van netwerkcomponenten zoals switches en routers, zodat hun machines correct kunnen functioneren. Wanneer een productiesite veel verschillende machines van uiteenlopende fabrikanten bevat, ontstaat al snel een situatie waarbij een groot aantal afzonderlijke private netwerken naast elkaar bestaat. Omdat deze netwerken zelden op elkaar worden afgestemd of gestandaardiseerd, leidt dit tot een uiterst gefragmenteerde netwerkstructuur. Hierdoor wordt het uniform beheren van de netwerken op verschillende VPK-sites vrijwel onmogelijk, aangezien elke site een unieke combinatie van leveranciers, machines en netwerktopologieën bevat.
Daarnaast vereisen veel fabrikanten een externe verbinding met hun machines om op afstand onderhoud te kunnen uitvoeren, real-time monitoring mogelijk te maken of software-updates te installeren. Dit betekent dat er op de productiesite van Alizay op twee mogelijke manieren een connectie met het internet kan worden gerealiseerd. De eerste optie is dat de fabrikant gebruikmaakt van het bestaande bekabelde netwerk van de site, dat wordt beheerd door VPK. In dit geval moet er bij de configuratie van de firewallregels rekening mee worden gehouden dat dit verkeer niet per ongeluk wordt geblokkeerd, anders verliest de fabrikant de mogelijkheid om zijn machines op afstand te beheren.
De tweede optie is dat de fabrikant een 4G-router met een eigen SIM-kaart gebruikt, zodat er geen directe verbinding met het netwerk van VPK nodig is. Hoewel dit in sommige gevallen een veiliger alternatief lijkt, introduceert het ook uitdagingen op het gebied van toezicht en controle over externe verbindingen.
Beide methoden brengen aanzienlijke beveiligingsrisico’s met zich mee. Veel van de private netwerken die door machinefabrikanten worden opgezet, voldoen niet aan de meest recente cybersecuritystandaarden. Dit betekent dat ze kwetsbaar kunnen zijn voor aanvallen van buitenaf. Indien een hacker erin slaagt toegang te krijgen tot zo’n slecht beveiligd netwerk, kan dit een potentiële ingang vormen naar het bredere IT- en OT-netwerk van de productiesite. Dit zou ernstige gevolgen kunnen hebben, zoals productieonderbrekingen, verlies van kritieke data en zelfs de manipulatie van industriële processen.

















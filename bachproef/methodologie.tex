%%=============================================================================
%% Methodologie
%%=============================================================================

\chapter{\IfLanguageName{dutch}{Methodologie}{Methodology}}%
\label{ch:methodologie}

%% TODO: In dit hoofstuk geef je een korte toelichting over hoe je te werk bent
%% gegaan. Verdeel je onderzoek in grote fasen, en licht in elke fase toe wat
%% de doelstelling was, welke deliverables daar uit gekomen zijn, en welke
%% onderzoeksmethoden je daarbij toegepast hebt. Verantwoord waarom je
%% op deze manier te werk gegaan bent.
%% 
%% Voorbeelden van zulke fasen zijn: literatuurstudie, opstellen van een
%% requirements-analyse, opstellen long-list (bij vergelijkende studie),
%% selectie van geschikte tools (bij vergelijkende studie, "short-list"),
%% opzetten testopstelling/PoC, uitvoeren testen en verzamelen
%% van resultaten, analyse van resultaten, ...
%%
%% !!!!! LET OP !!!!!
%%
%% Het is uitdrukkelijk NIET de bedoeling dat je het grootste deel van de corpus
%% van je bachelorproef in dit hoofstuk verwerkt! Dit hoofdstuk is eerder een
%% kort overzicht van je plan van aanpak.
%%
%% Maak voor elke fase (behalve het literatuuronderzoek) een NIEUW HOOFDSTUK aan
%% en geef het een gepaste titel.



De eerste fase zal gewijd worden aan het analyseren van de concrete problemen die zich mogelijks kunnen voordoen bij het behouden van de originele firewall, dit kan gaan over technische en non-technische aspecten die mogelijks de dienstverlening binnen de productie kan verstoren en de financiële schade die VPK daar mogelijk zou kunnen door oplopen. Ook zal er worden gekeken hoe het implementeren van die nieuwe firewall toepassing dit probleem zouden kunnen beheren en voorkomen. Hiervoor zal er gebruik gemaakt worden van bedrijfsinterne knowledge bases, diverse papers en studies met betrekking tot netwerkbeveiliging. Zo zal er een beter inzicht te verkrijgen zijn over de mogelijke firewall opstellingen en welke strategieën er kunnen worden toegepast voor het beschermen van het ICS met behulp van firewalltoepassingen. Ook zal er worden gekeken naar andere aspecten die mogelijks spelen bij het maken van een doordachte keuze. Dit kan gaan over de ervaringen van het huidige netwerkteam met een bepaalde vendor. Maar mogelijks over de integratie met andere netwerkcomponenten en de mogelijkheid om de firewall centraal te beheren. De geschatte duurtijd van deze fase bedraagt drie dagen die verspreid zullen zijn over drie weken.


Tijdens de tweede fase zal er een interne bevraging zijn bij systeembeheerders en netwerkprofessionals op bedrijfsniveau. Hun jarenlange ervaring omtrent netwerkbeveiliging en het optimaliseren van talloze beveiligingsmaatregelen binnen VPK kan kostbare inzichten opleveren voor het achterhalen van de huidige netwerkopstelling van het bedrijf. Deze informatie zal gebruikt kunnen worden voor het configureren van de firewall zodanig dat deze optimaal geconfigureerd is in het netwerk. Zo kunnen we de schade die een aanval zou kunnen aanrichten zo minimaal mogelijk houden en zal de aanval de dienstverlening zo weinig mogelijk verstoren. Ook kan deze informatie nuttig zijn voor de evaluatie in de vijfde fase. De geschatte duurtijd van deze fase bedraagt drie dagen die verspreid zullen zijn over drie weken.

In de derde fase verwerken we de grote hoeveelheid verzamelde data om een helder overzicht te creëren. Dit maakt het makkelijker om te beslissen of we de huidige firewall opnieuw configureren of een volledig nieuwe implementeren. Deze keuze baseren we op verschillende analyses, zoals een SWOT en RASCI analyse. Hiermee brengen we niet alleen de technische voor- en nadelen in kaart, maar kijken we ook naar welke stakeholders welke taken op zich nemen na de implementatie. Op basis van deze inzichten stellen we een concreet plan op voor eventuele aanpassingen aan de firewall en het netwerk van de productiesite. De geschatte duurtijd van deze fase bedraagt twee dagen die verspreid zullen zijn over twee weken.

In het vierde deel zal er worden toegespitst op het perfectioneren van beveiligingstoepassingen. Hierbij zal er met behulp van het plan dat is opgesteld in de derde fase de huidige firewalltoepassing geherconfigureerd worden. Als er aan de hand van de verschillende analyses blijkt dat een andere firewall dan de huidige firewall een betere 'fit' zou kunnen zijn voor de eisen en noden van VPK dan kan er worden gekozen om gebruik te maken van een compleet nieuwe firewall die ook betere bescherming zal bieden tegen aanvallen op het ICS. Ook zullen er aanpassingen kunnen worden aangebracht aan de rest van het netwerk die de firewall zou kunnen helpen bij het tegenhouden van aanvallen op de ICS. De geschatte duurtijd van deze fase bedraagt drie dagen die verspreid zullen zijn over drie weken.


De vijfde en tevens laatste fase zal bestaan uit het evalueren van de impact van de geherconfigureerde firewall en de voorgestelde aanpassingen aan het netwerk. Er zullen een reeks criteria worden opgesteld waaraan voldaan moet zijn. Deze criteria zullen er voor zorgen dat we duidelijk kunnen aantonen dat mogelijke wijzigingen die we hebben aangebracht het makkelijker maken voor de systeembeheerders om de firewall te beheren en een ze ook een duidelijker inzicht krijgen van welke zaken er juist gebeuren aan de interne kant van de firewall. Daarnaast zal worden beschreven welke specifieke maatregelen het beste kunnen worden genomen om het netwerk en de firewall van het productiebedrijf beter te beschermen tegen aanvallen op de ICS. De geschatte duur van deze fase bedraagt twee dagen, verdeeld over twee weken.  




%%=============================================================================
%% Conclusie
%%=============================================================================

\chapter{Conclusie}%
\label{ch:conclusie}

% TODO: Trek een duidelijke conclusie, in de vorm van een antwoord op de
% onderzoeksvra(a)g(en). Wat was jouw bijdrage aan het onderzoeksdomein en
% hoe biedt dit meerwaarde aan het vakgebied/doelgroep? 
% Reflecteer kritisch over het resultaat. In Engelse teksten wordt deze sectie
% ``Discussion'' genoemd. Had je deze uitkomst verwacht? Zijn er zaken die nog
% niet duidelijk zijn?
% Heeft het onderzoek geleid tot nieuwe vragen die uitnodigen tot verder 
%onderzoek?



Tijdens het onderzoek werdt er gezocht naar de meest geschikte firewalloplossing en algemene netwerkopstelling voor de productiesite van Alizay. De centrale onderzoeksvraag was of de huidige infrastructuur, gebaseerd op een Sophos firewall, behouden moet blijven, of dat het interessanter is om over te stappen naar een uniforme aanpak zoals die al bij andere VPK-sites wordt toegepast, met uitsluitend Palo Alto firewalls.

\section{Resultaten van het onderzoek}
Het vergelijken van beide opties was niet eenvoudig, aangezien beide oplossingen hun eigen voor en nadelen hebben. De huidige Sophos configuratie heeft als voordeel dat deze reeds aanwezig en operationeel is, wat betekent dat er geen grote veranderingen nodig zijn. Dit verkleint het risico op tijdelijke storingen of downtime en vergt minder inzet van het IT team op korte termijn. Anderzijds zijn er bij deze oplossing enkele belangrijke nadelen. De Sophos firewall is minder efficiënt om te beheren en biedt minder stabiliteit en veiligheid in vergelijking met de Palo Alto oplossingen.
De Palo Alto firewalls die op andere sites van VPK al worden gebruikt, hebben een betere reputatie op vlak van betrouwbaarheid, prestaties en schaalbaarheid. Ze bieden bovendien geavanceerdere beveiligingsfuncties en een meer gestroomlijnd beheer, vooral wanneer er op meerdere locaties met dezelfde technologie wordt gewerkt. De overstap naar een oplossing waarbij er gebruik wordt gemaakt van een Palo Alto firewall pair vraagt echter wel om grootschalige aanpassingen aan de infrastructuur en kan leiden tot tijdelijke hinder of downtime.


\section{Aanbeveling}
Hoewel beide keuzes mogelijk zijn, wijst de analyse op langere termijn toch in de richting van een overstap naar Palo Alto firewalls. Deze oplossing zorgt voor meer uniformiteit binnen het hele VPK netwerk, wat het beheer makkelijker maakt en toekomstige aanpassingen makkelijker maakt. De verbeterde beveiliging en stabiliteit maken de initiële investering in tijd en middelen de moeite waard.
Het is aan te raden om deze overstap zorgvuldig te plannen en gefaseerd uit te voeren, bijvoorbeeld tijdens geplande onderhoudsperiodes, om de impact op de productie tot een minimum te beperken. Zo kan men er voor zorgen dat het netwerk beter zal functioneren op alle vlakken zonder dat de werking van de productie in gevaar komt.




%---------- Inleiding ---------------------------------------------------------

% TODO: Is dit voorstel gebaseerd op een paper van Research Methods die je
% vorig jaar hebt ingediend? Heb je daarbij eventueel samengewerkt met een
% andere student?
% Zo ja, haal dan de tekst hieronder uit commentaar en pas aan.

%\paragraph{Opmerking}

% Dit voorstel is gebaseerd op het onderzoeksvoorstel dat werd geschreven in het
% kader van het vak Research Methods dat ik (vorig/dit) academiejaar heb
% uitgewerkt (met medesturent VOORNAAM NAAM als mede-auteur).
% 

\section{Inleiding}%
\label{sec:inleiding}

Door de toenemende complexiteit en frequentie van cyberaanvallen is het steeds moeilijker en noodzakelijker voor bedrijven om zich te beschermen tegen deze dreigingen \autocite{saravanan2019}. Het is daarom uitermate belangrijk dat bepaalde organisaties zich hiertegen gaan wapenen met effectieve maatregelen op het vlak van cybersecurity. In dit onderzoek zal de focus liggen op het analyseren en evalueren van verschillende firewalltoepassingen en de impact op het beveiligen van de ICS van het productiebedrijf.


Volgens \textcite{pan2017} is cybersecurity een hoge prioriteit bij veel bedrijven, vooral door het groeiende aantal machines en systemen die op één of andere manier verbonden zijn met het internet om te communiceren met onder meer ERP-systemen.  Volgens \textcite{Lin2017} worden ICS'en op grote schaal gebruikt in verschillende kritieke infrastructuren van onder andere de olie-, water- en elektriciteitsindustrie. In het verleden beschikten de meeste van deze ICS'en niet over authentificatie- en versleutelingsmechanismen, waardoor ze kwetsbaar waren voor aanvallen door hackers. Dit is een zwak punt in het netwerk van een productiebedrijf die gebruik maken van een ICS. Zonder een performante firewall zijn deze systemen zeer kwetsbaar voor allerhande cyberbedreigingen, met name deze op de ICS. Daarom is de keuze voor een geschikte IDS- en IPS-firewalltoepassing belangrijk voor het beschermen van bedrijfsnetwerken tegen mogelijke aanvallen van buitenaf die schade kunnen aanrichten aan de industriële infrastructuur van het bedrijf. Door deze aanvallen kan de dienstverlening van het bedrijf op grote schaal verstoord worden. Volgens \textcite{Nwanya2017} zou het bedrijf aanzienlijk grote verliezen kunnen lijden door de stilstand van de productie. Het doel van dit onderzoek is dan ook om systeembeheerders en IT-professionals binnen het productiebedrijf gerichtere informatie te geven over hoe ze zich het beste beschermen tegen aanvallen op de ICS. Deze informatie helpt hen bij het uitkiezen van de meest effectieve firewalltoepassing, zodat zij de beveiliging van hun IT-infrastructuur kunnen waarborgen en de continue werking van de productie kunnen garanderen. Hierdoor kunnen de bevindingen van dit onderzoek direct worden toegepast in relevante cases in het productiebedrijf.

Daarom zal dit onderzoek het probleem aankaarten dat een bedrijf heeft bij het kiezen van de juiste firewalltoepassing die haar infrastructuur zal beschermen tegen aanvallen op het ICS. Dit kan verschillen van bedrijf tot bedrijf. Daarom is een gepaste onderzoeksvraag voor deze bachelorproef: ``\texttt{Welke firewalltoepassingen zal het beste ICS-aanvallen tegengaan binnen VPK Packaging Group?}'' Enkele deelvragen kunnen zijn: ``\texttt{Welke zwaktes in de huidige IT/OT architectuur maken VPK kwetsbaar voor ICS-aanvallen?}'', ``\texttt{Welke soorten firewall systemen worden er momenteel gebruikt binnen het bedrijf?}'', ``\texttt{Welke kosten en onderhoudsvereisten zijn verbonden aan de implementatie van een nieuwe firewall?}'' ``\texttt{Welke specifieke bedreigingen heeft het bedrijf eerder ervaren of verwacht het te ervaren?}'', ``\texttt{Hoe kunnen verschillende firewalltoepassingen worden geëvalueerd op hun effectiviteit bij het voorkomen van ICS-aanvallen binnen VPK Packaging Group?
}'' en  ``\texttt{Hoe kan de gekozen firewalltoepassing optimaal worden geïmplementeerd en geconfigureerd binnen de bestaande IT/OT-infrastructuur van VPK Packaging Group?}'' Dit onderzoek zal beperkt blijven tot het beveiligen van de ICS van het bedrijf. Mogelijks zullen door aanpassingen aan de infrastructuur voor de optimalisatie van de bescherming van de ICS ook andere delen van het netwerk beter beschermd zijn. Echter is dit niet het hoofddoel. Het doel van dit onderzoek is ook om betere inzichten te krijgen in de trend van de verschillende soorten firewalltoepassingen en de verschillende use cases in de praktijk. Dit doel zal bereikt worden door een uitgebreide literatuurstudie en het analyseren van deze casestudy binnen een specifiek productiebedrijf. En daarna deze kennis toe te passen om het ICS van het bedrijf beter te beschermen tegen aanvallen.

Dit alles zal worden samengevat in een rapport waarin een groot aantal aanbevelingen staan voor leden van het cybersecurityteam binnen het productiebedrijf. Hierdoor zullen zij beter in staat zijn om hun infrastructuur te beschermen tegen ICS-cyberaanvallen van buitenaf. Ook zullen zij bedreigingen beter kunnen identificeren, en hierdoor veilige en effectieve firewalltoepassingen kunnen implementeren. Daardoor zal het doel van deze bachelorproef ook bereikt zijn en zal VPK beter beschermd zijn tegen de steeds gevaarlijkere en grootschaligere ICS-dreigingen. Dit zorgt ervoor dat de downtime van machines en de integriteit van data binnen het bedrijf steeds gewaarborgd blijven. Deze onderzoeksdoelstelling is specifiek, meetbaar, acceptabel, relevant en tijdsgebonden (SMART) en zal daardoor resulteren in een stevige set van aanbevelingen voor betere cybersecurity practices.



\section{Literatuurstudie}%
\label{sec:literatuurstudie}

Firewalls bestaan al sinds de jaren' 80, toendertijd deden firewalls slechts enkel aan basic packet filtering. Sindsdien zijn firewalls steeds blijven evolueren, enerzijds door de groeiende digitalisering en anderzijds door de grotere cyberdreigingen tegen industriële infrastructuur van productiebedrijven \autocite{Wusteney2021}. Vandaag de dag wordt er gebruik gemaakt van Next Generation Firewalls (NGFW’s) die diepe packet inspection uitvoeren en op basis hiervan zullen beslissen of een packet wordt geblokkeerd of niet. Sedert 2020 maken deze NGFW’s ook gebruik van Artificiële Intelligentie (AI). \autocite{Ahmadi2023}. Firewalls die gebruik maken van AI genereren beveiligingsmaatregelen en dwingen deze af op basis van het continue netwerkverkeer, waardoor de blootstelling aan nieuwe bedreigingen aanzienlijk wordt verminderd. \autocite{PaloAltoFW2024}

Ondanks de snelle evolutie van firewalls blijft een groot struikelblok bij het toepassen hiervan de complexiteit van bedrijfsnetwerken. Volgens \textcite{Bringhenti2023} wordt het configureren van security functies typisch nog steeds manueel gedaan. Maar omdat moderne netwerken zodanig complex en dynamisch zijn, is de manuele configuratie hiervan niet haalbaar. Dit zorgt ervoor dat het kiezen van een gepaste firewall niet alleen draait om de manier waarop mogelijke threats worden afgehandeld, maar ook de manier waarop de automatisatie en configuratie zullen kunnen worden toegepast. Uit diezelfde bron blijkt ook dat dit ervoor zorgt dat voornamelijk bedrijfsnetwerken met een groot en complex netwerk kwetsbaar zijn voor cyber aanvallen. Echter is er een manier om de complexiteit van een groot bedrijfsnetwerk te verminderen door het toepassen van netwerk segmentatie \autocite{Farooq2023}. Hierdoor wordt het schrijven van veiligheidsmaatregelen doenbaar en kunnen er robuuste security policies worden opgesteld voor het netwerk. 

Om netwerksegmentatie correct toe te passen zijn geavanceerde oplossingen zoals die van Palo Alto Networks essentieel. Volgens het National Institute of Standards \& Technology (NIST) is Palo Alto één van de marktleiders op het vlak van cybersecurity toepassingen, waaronder NGFW’s, cloud-based security services, advanced endpoint protection en threat intelligence. \autocite{TechnicalWhitepaper2014} Ook heeft Palo Alto een cloud based malware protection engine met de naam WildFire. WildFire is een IDS die files die gecategoriseerd zijn als mogelijke dreiging gaan uitvoeren in een sandbox omgeving. In deze sandbox worden de dreigingen ook geanalyseerd met behulp van machine learning algoritmes die gedrag en patronen die indicatief zijn voor kwaadaardige activiteit zullen detecteren. \autocite{PaloAltoWF2024}. Volgens \textcite{Faizan2019} zijn naast Palo Alto ook Cisco ASA en Fortinet Fortigate belangrijke NGFW's die gebruik maken van IPS- en IDS features. In diezelfde bron staat ook dat Fortinet optimaal presteert binnen het Fortinet Fabric-netwerk, maar dit zorgt ervoor dat bedrijven beperkt zijn tot het gebruiken van Fortinet apparatuur. Apparaten van andere leveranciers verminderen de prestaties. Cisco biedt stand-alone apparatuur die ook performant kan presteren met verschillende netwerkapparaten van andere merken, maar is gemiddeld duurder dan Fortinet. 

Ook werd er uit een studie van \textcite{Skybakmoen2018} de Total Cost of Ownership (TCO) per beschermde megabit per seconde (Mbps) voor firewalltoepassingen berekend. Cisco had een TCO van \$28 per Mbps en een beveiligingseffectiviteit van 71,8\%. Palo Alto Networks had een TCO van \$7 per Mbps en een beveiligingseffectiviteit van 98,7\%. Fortinet bleek het meest kostenefficiënt met een TCO van \$2 per Mbps en een beveiligingseffectiviteit van 99,3\%.


\section{Methodologie}%
\label{sec:methodologie}

De eerste fase zal gewijd worden aan het analyseren van de concrete problemen die zich voordoen bij een mogelijke aanval op een ICS in een specifiek bedrijf, dit zou kunnen gaan over de verstoring van de dienstverlening binnen de productie en de financiële schade die het bedrijf kan oplopen. En hoe firewalltoepassingen dit probleem zouden kunnen beheren en voorkomen. Hiervoor zal er gebruik gemaakt worden van bedrijfsinterne knowledge bases, diverse papers en studies met betrekking tot netwerkbeveiliging. Er zal worden gekeken naar vorige ICS aanvallen binnen het bedrijf en welke strategieën en technieken de aanvallers hiervoor gebruikten. Zo zal er een beter inzicht te verkrijgen zijn over de mogelijke vormen van ICS aanvallen en welke strategieën er kunnen worden toegepast voor het beschermen van het ICS met behulp van firewalltoepassingen. Ook zal er worden gekeken naar ander soorten van ICS aanvallen die nog niet zijn uitgevoerd binnen VPK. Hierdoor zal VPK zich ook beschermen tegen ongeziene aanvallen. Dit zal resulteren in een checklist van mogelijke specifieke acties die kunnen worden genomen voor een betere bescherming van het ICS van het productiebedrijf. De geschatte duurtijd van deze fase bedraagt drie dagen die verspreid zullen zijn over drie weken.

Tijdens de tweede fase zal er een interne bevraging zijn bij systeembeheerders en netwerkprofessionals op bedrijfsniveau. Hun jarenlange ervaring omtrent netwerkbeveiliging en het optimaliseren van talloze beveiligingsmaatregelen binnen VPK kan kostbare inzichten opleveren voor het achterhalen van de huidige netwerkopstelling van het bedrijf. Deze informatie zal gebruikt kunnen worden voor het configureren van de firewall zodanig dat deze zo optimaal mogelijk geconfigureerd is in het netwerk. Zo kunnen we de schade die een aanval zou kunnen aanrichten zo minimaal mogelijk houden en zal de aanval de dienstverlening zo weinig mogelijk verstoren. Ook kan deze informatie nuttig zijn voor de evaluatie in de vijfde fase. De geschatte duurtijd van deze fase bedraagt drie dagen die verspreid zullen zijn over drie weken.

In de derde fase zullen er simulaties worden uitgevoerd om de zwaktes van de huidge firewalltoepassing tegenover ICS aanvallen in te schatten. Deze simulaties zijn noodzakelijk voor het inschatten van de huidige situatie, uit deze reeks van geautomatiseerde pen tests zal blijken over bepaalde zwaktes aanwezig zijn die mogelijks zullen moeten verholpen worden in de volgende fase. Er zal er op basis van data van vorige aanvallen kunnen worden gekeken welke technieken het meeste gebruikt worden door aanvallers om toegang tot een systeem te verkrijgen. Deze data zal worden gebruikt om gecontroleerde simulaties  uit te voeren van gelijkaardige aanvallen op het huidige netwerk. Op basis van deze inzichten zal het mogelijk zijn om een concreet plan op te stellen over welke zaken er juist moeten worden aangepast aan de firewall en eventueel aan het netwerk van het productiebedrijf. De geschatte duurtijd van deze fase bedraagt twee dagen die verspreid zullen zijn over twee weken.

In het vierde deel zal er worden toegespitst op het perfectioneren van beveiligingstoepassingen. Hierbij zal er met behulp van het plan dat is opgesteld in de derde fase de huidige firewalltoepassing geherconfigureerd worden. Als er bij een vergelijkende studie blijkt dat een andere firewall dan de huidige firewall een betere 'fit' zou kunnen zijn voor de eisen en noden van VPK dan kan er worden gekozen om gebruik te maken van een compleet nieuwe firewall die ook betere bescherming zal bieden tegen aanvallen op het ICS. Ook zullen er aanpassingen kunnen worden aangebracht aan de rest van het netwerk die de firewall zou kunnen helpen bij het tegenhouden van aanvallen op de ICS. De geschatte duurtijd van deze fase bedraagt drie dagen die verspreid zullen zijn over drie weken.

De vijfde en tevens laatste fase zal bestaan uit het evalueren van de impact van de geherconfigureerde firewall en de voorgestelde aanpassingen aan het netwerk. De criteria voor deze evaluatie omvatten de effectiviteit van de firewall- en netwerkaanpassingen ten opzichte van de initiële situatie van het netwerk en de firewall, het vermogen van de firewall om een reeks vooraf opgestelde aanvallen die gebruik maken van specifieke technieken op de productie-infrastructuur af te weren, en de snelheid van de firewall bij het detecteren van bedreigingen. Daarnaast zal worden beschreven welke specifieke maatregelen het beste kunnen worden genomen om het netwerk en de firewall van het productiebedrijf beter te beschermen tegen aanvallen op de ICS. De geschatte duur van deze fase bedraagt twee dagen, verdeeld over twee weken. 

\section{Verwachte resultaten}%
\label{sec:verwachte-resultaten}

Er wordt verwacht aan te tonen dat firewalltoepassingen essentieel zijn voor de beveiliging van een ICS in het productiebedrijf. Aangezien er geen universele oplossing bestaat, zullen specifieke aanbevelingen worden gecreëerd voor het selecteren en implementeren van firewalls die optimaal presteren binnen de context van die bepaalde productieomgeving. Dit zal zorgen voor een verbeterde bescherming tegen ICS-aanvallen om zo de minimale hoeveelheid onderbrekingen in de productieprocessen te kunnen garanderen. Deze inzichten zullen waardevol zijn voor IT-professionals die verantwoordelijk zijn voor de netwerkbeveiliging in hun productiebedrijf.

\section{Verwacht resultaat, conclusie}%
\label{sec:verwachte_resultaten}

Zoals al eerder besproken zal dit onderzoek een grote meerwaarde zijn voor IT-professionals die instaan voor het implementeren en onderhouden van de firewalltoepassingen die het bedrijf beveiligen tegen ICS aanvallen. Door dit onderzoek zullen zij in staat zijn om een geschikte firewalltoepassing te kiezen die gepast is binnen hun use case. Dit zorgt ervoor dat hun netwerk beter beveiligd zal zijn. Dit biedt niet alleen een meerwaarde op het vlak van IT, maar kan ook gunstig zijn voor de algemene werking van het bedrijf. Men zal hoe dan ook niet kunnen negeren dat het gebruiken van een firewall een gunstig effect zal hebben op de capaciteit van het netwerk om ICS aanvallen af te weren


